\documentclass[]{article}
\usepackage{lmodern}
\usepackage{amssymb,amsmath}
\usepackage{ifxetex,ifluatex}
\usepackage{fixltx2e} % provides \textsubscript
\ifnum 0\ifxetex 1\fi\ifluatex 1\fi=0 % if pdftex
  \usepackage[T1]{fontenc}
  \usepackage[utf8]{inputenc}
\else % if luatex or xelatex
  \ifxetex
    \usepackage{mathspec}
  \else
    \usepackage{fontspec}
  \fi
  \defaultfontfeatures{Ligatures=TeX,Scale=MatchLowercase}
\fi
% use upquote if available, for straight quotes in verbatim environments
\IfFileExists{upquote.sty}{\usepackage{upquote}}{}
% use microtype if available
\IfFileExists{microtype.sty}{%
\usepackage{microtype}
\UseMicrotypeSet[protrusion]{basicmath} % disable protrusion for tt fonts
}{}
\usepackage[margin=1in]{geometry}
\usepackage{hyperref}
\hypersetup{unicode=true,
            pdftitle={a},
            pdfborder={0 0 0},
            breaklinks=true}
\urlstyle{same}  % don't use monospace font for urls
\usepackage{graphicx,grffile}
\makeatletter
\def\maxwidth{\ifdim\Gin@nat@width>\linewidth\linewidth\else\Gin@nat@width\fi}
\def\maxheight{\ifdim\Gin@nat@height>\textheight\textheight\else\Gin@nat@height\fi}
\makeatother
% Scale images if necessary, so that they will not overflow the page
% margins by default, and it is still possible to overwrite the defaults
% using explicit options in \includegraphics[width, height, ...]{}
\setkeys{Gin}{width=\maxwidth,height=\maxheight,keepaspectratio}
\IfFileExists{parskip.sty}{%
\usepackage{parskip}
}{% else
\setlength{\parindent}{0pt}
\setlength{\parskip}{6pt plus 2pt minus 1pt}
}
\setlength{\emergencystretch}{3em}  % prevent overfull lines
\providecommand{\tightlist}{%
  \setlength{\itemsep}{0pt}\setlength{\parskip}{0pt}}
\setcounter{secnumdepth}{0}
% Redefines (sub)paragraphs to behave more like sections
\ifx\paragraph\undefined\else
\let\oldparagraph\paragraph
\renewcommand{\paragraph}[1]{\oldparagraph{#1}\mbox{}}
\fi
\ifx\subparagraph\undefined\else
\let\oldsubparagraph\subparagraph
\renewcommand{\subparagraph}[1]{\oldsubparagraph{#1}\mbox{}}
\fi

%%% Use protect on footnotes to avoid problems with footnotes in titles
\let\rmarkdownfootnote\footnote%
\def\footnote{\protect\rmarkdownfootnote}

%%% Change title format to be more compact
\usepackage{titling}

% Create subtitle command for use in maketitle
\newcommand{\subtitle}[1]{
  \posttitle{
    \begin{center}\large#1\end{center}
    }
}

\setlength{\droptitle}{-2em}

  \title{a}
    \pretitle{\vspace{\droptitle}\centering\huge}
  \posttitle{\par}
    \author{}
    \preauthor{}\postauthor{}
    \date{}
    \predate{}\postdate{}
  

\begin{document}
\maketitle

Analyzing the Efficacy of Flow Restrictor Plates in Denil Fishways for
Passage of Arctic Grayling Tyler Blue Habitat connectivity is imperative
in the preservation of access to habitat, food, and refuge for fish and
other aquatic species. The Arctic Grayling (Thymallus Arcticus) is a
population of fish that has seen the impact that cutting off upstream
habitat has on the life cycle of a species. A species once abundant in
the rivers and lakes of Michigan and Montana, the Arctic Grayling now
resides only in the Big Hole River watershed. To help preserve the
Arctic Grayling, 63 Denil fishways have been installed on irrigation
diversion structures throughout this watershed. However, this style of
fish ladder requires higher flow levels to facilitate passage, but
irrigators need every bit of water they can get. In an attempt to meet
the need of Arctic grayling and irrigators, a study was conducted to
test the efficacy of flow-control weirs (‰ÛÏflow-restrictor plates‰Û)
installed at the upstream end of the Denil fishway to decrease the
required flowrate needed to pass Arctic grayling. This study
investigated three distinct restrictor plate designs in addition to a
control (no plate installed). Each treatment received the same five
trials in order to better compare flow restriction, passage efficiency,
and several other hydraulic and ecologic factors. Upon the completion of
the data collection phase, analysis was done to assess the efficacy of
installing any of these plates to the upstream end of the Denil fishway.
The end goal was to develop a model that could predict passage
efficiency in a Denil fishway for a given plate based on a set of other
variables. After several iterations, it was found that the sole
predictor of passage efficiency will be the amount of water running
through the fish ladder. Work will continue to develop a standard for
the installation of these flow control weirs, but at this time, it
appears that these plates may in fact inhibit passage.


\end{document}
