\documentclass[a4paper]{article}
%\usepackage{simplemargins}

%\usepackage[square]{natbib}
\usepackage{amsmath}
\usepackage{amsfonts}
\usepackage{amssymb}
\usepackage{graphicx}

\usepackage{Sweave}
\begin{document}
\Sconcordance{concordance:water-center_test.tex:water-center_test.Rnw:%
1 9 1 1 0 3 1 1 34 1 25 3 1 1 4 53 0 1 2 4 1 1 4 6 0 1 2 4 1 1 6 103 0 %
1 2}

\pagenumbering{gobble}



\Large
 \begin{center}

\begin{Schunk}
\begin{Soutput}
Convective suppression before and during the 2017 northern Great Plains flash drought: Implications for forecasting 
Testing registration 
The Surface Water Assessment and Monitoring Program (SWAMP) and the MBMG Data Center 
Assessing Connectivity Benefits of Denil Fishways, an Integrated Approach 
A Groundwater Model for the Meadow Village Alluvial Aquifer, Big Sky, Gallatin County, Montana 
Using groundwater models to explore how beaver-mimicry stream restoration affects dynamic seasonal water storage 
The Confluence of Law and Science: A collaborative effort to manage water resources on the Teton River, Montana. 
State-Wide Groundwater Monitoring Network ‰ÛÒ  Assessing Drought Impacts and Ground Truthing Big Data 
Incorporating Geophysical Methods with Hydrogeologic Studies to Investigate Late-summer Dewatering in Lolo Creek, Southwest Missoula, Montana 
Epilithic Biomass Abundance and Composition: Influences on Allochthonous and Autochthonous Nitrogen Sources 
Influence of large wood on sediment routing in a mixed bedrock-alluvial stream 
Leveraging DNA to Identify Sources of Fecal Pollution in Stormwater 
Application of Dimensionless Sediment Rating Curves to Predict Suspended-Sediment Concentrations, Bedload, and Annual Sediment Loads for Rivers and Streams 
Non-resident selenium imports to Lake Koocanusa and Bighorn Lake, MT: Sources, biogeochemical cycling, and tailwater implications 
Geochemistry of metals and nutrients in fine-sediment pore water in Blacktail and Silver Bow creeks, Butte, Montana 
Convective suppression before and during the flash drought of 2017 
H20 Tools: An Online Platform for Collaboration and Data Integration 
Increased drought intensity driven by warming in the upper Missouri river basin 
Understanding the relation between energy and water resources in the Williston Basin 
Forest Service NHD Stewardship Strategy 
Pollutant Transport via Sewage 
Application of the Wetted Perimeter Methodology to Identify and Mitigate Potential Impacts from Proposed Exploratory Drilling- Iron Creek, Beartooth R.D., Custer Gallatin National Forest 
Little Bitterroot Lake Association - 20 Years of 
Effectiveness of Montana's Streamside Management Zone Law at Protecting Stream Temperatures during Forest Harvesting 
Biogeochemical dynamics in a wetland-stream continuum 
Comparison of  Satellite-Based,  Remote-Sensing Methods for Estimating Evapotranspiration from Irrigated Land in the Flathead and Smith River Basins of Montana. 
Mapping, Managing, and Maintaining data with Survey123 and ArcGis Online 
Funding Projects that Benefit Natural Resources 
"Please Pass the Boreal Toad": Aquatic Organism Passage and Stream Restoration on South Clear Creek Colorado 
"Please Pass the Boreal Toad": Aquatic Organism Passage and Stream Restoration on South Clear Creek Colorado 
Consortium for Research on Environmental Water Systems (CREWS): a Montana collaborative effort for integrated research and application 
Exploring the size of contaminants in wastewater 
Collaborative Science: A Case Study on Mine Reclamation Activities to Improve Ecological Health of Yellowstone‰Ûªs Soda Butte Creek 
The Big Levers - Management, Soils, and Weather. Interactions controlling soil water and nitrate loss in a non-irrigated cropping system 
Modification of Reverse Osmosis Membrane Surface with Nisin-Coated Silica Particles to Mitigate Biofouling 
Passive Evaporation Enhancement of Acidic Mine Water 
Using Detection Dogs to Monitor Environmental Contaminants in Freshwater Ecosystems Via Sentinel Species 
Using sediment maps to understand fluvial sediment dynamics associated with in-channel, deformable structures 
Pre-restoration characteristics of high elevation mesic sites in greater sage-grouse habitats 
Emerging contaminant increases wetland methane fluxes by stimulating production and potential trophic cascade 
Living Filtration Membranes 
Living Filtration Membranes 
Reconnection and restoration of Elbow Coulee in the Sun River watershed 
Host rock influences metal abundance and speciation in hot springs 
Using groundwater modeling to assess groundwater and stream connectivity in a river restoration application 
Aqueous metal and semimetal speciation in Silver Bow and Blacktail Creeks 
From Droughts to Floods and Back Again: Looking back at Montana 2017 ‰ÛÒ 2018 Streamflows 
Using longitudinal synoptics of water quality along Hyalite Creek and the Gallatin Valley to understand the distribution of groundwater sources to stream flow generation in the Gallatin River Watershed 
Nonpoint Source Total Maximum Daily Load Implementation  - Example Watershed-Scale Evaluations 
Analyzing the Efficacy of Flow Restrictor Plates in Denil Fishways for Passage of Arctic Grayling 
\end{Soutput}
\end{Schunk}


\hspace{10pt}
% Author names
\large
\begin{Schunk}
\begin{Sinput}
> for (i in 1:nrow(meld)){
+ names[[i]][1:length(names[i])]
+ }
\end{Sinput}
\end{Schunk}

\hspace{10pt}
\end{center}
\hspace{10pt}
\normalsize
\begin{Schunk}
\begin{Soutput}
13
\newpage
46
\newpage
38
\newpage
25
\newpage
35
\newpage
5
\newpage
33
\newpage
41
\newpage
28
\newpage
11
\newpage
26
\newpage
34
\newpage
8
\newpage
2
\newpage
21
\newpage
14
\newpage
17
\newpage
37
\newpage
45
\newpage
39
\newpage
19
\newpage
12
\newpage
40
\newpage
4
\newpage
27
\newpage
10
\newpage
48
\newpage
42
\newpage
43
\newpage
43
\newpage
3
\newpage
6
\newpage
7
\newpage
22
\newpage
31
\newpage
9
\newpage
47
\newpage
23
\newpage
36
\newpage
44
\newpage
30
\newpage
30
\newpage
1
\newpage
15
\newpage
20
\newpage
24
\newpage
16
\newpage
29
\newpage
32
\newpage
18
\newpage
\end{Soutput}
\end{Schunk}
\end{document}
